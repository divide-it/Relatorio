Essa seção detalha algumas questões levantadas durante a fase inicial do projeto.

\begin{itemize}
    \item Qual a arquitetura esperada da solução ?\\
    A arquitetura do projeto visa garantir uma implementação distribuída para a solução. Com um conceito que demanda por agrupamento de usuários, cálculos constantes de compras e divisão, gateways de pagamentos para transferências monetárias, esse é um projeto que pede por uma solução distribuída de seus elementos principais.
    
    \item Como implementar a comunicação entre a API e o app ?\\
O grupo decidiu usar a linguagem GRAPHQL~\cite{graphql}.

\item Quais features o usuário espera do App?\\
Autenticação, criar grupos de despesas, possibilidade de manipular os grupos, receber notificações pertinentes, poder atualizar o respectivo perfil e integração com formas de pagamento viáveis.

\item Como conheço quem são os participantes do grupo?\\
Possibilidade de consultar os membros de um grupo de despesas.

\item Quais dados devem ser trocados entre a API e o app ?\\
 Definição de interfaces de comunicação entre as partes, bem como modelos de dados que façam sentido para ambas as partes.

\item Como a API deve gerar notificações para o APP ?\\
Podemos usar um serviço (service worker) que roda em background no app e estabelece uma ponte de comunicação com a API (via socket) .

\item Como autorizar a emissão de pagamentos no nome do usuário em um método de sua escolha sem obrigá-lo a efetuar um segundo login ?\\
Criação do conceito de Convite, que permite a um usuário controlar a entrada e saída de outros usuários dentro de um grupo.

\item Como decidir quem de fato vai participar da divisão, sendo que um usuário pode receber a notificação mas não querer participar da divisão.\\
O usuário que recebe a notificação deve autorizar sua inclusão no grupo.

\item Como vou saber o score de quem está no grupo da divisão antes de confirmar que quero participar dela ?\\
Possibilidade de visualizar o score de um novo membro.

\end{itemize}